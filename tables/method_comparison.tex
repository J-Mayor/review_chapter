\begin{table*}
\centering
\caption{Comparison of Peculiar Velocity Methods}
\label{tab:method_comparison}
\begin{tabular}{lccccl}
\hline
Method & Galaxy Type & Precision & Systematics & Depth & Advantages/Disadvantages \\
       &             & (\%)      & Level       & (Mpc)  &                          \\
\hline
TF (optical) & Spirals & 15--25 & Medium & 150 & Large samples; dust, morphology \\
TF (NIR) & Spirals & 10--20 & Medium-Low & 150 & Less extinction; calibration \\
TF (HI) & Spirals & 15--25 & Medium & 200+ & Direct kinematics; HI mass limit \\
Baryonic TF & Spirals & 10--15 & Low & 150 & Tight relation; gas+stars needed \\
FP & Early-type & 10--20 & Medium & 200 & Complementary sample; velocity dispersion \\
SNe Ia & All types & 5--10 & Low & 500+ & High precision; sparse sampling \\
\hline
\end{tabular}
\tablecomments{Comparison of major peculiar velocity methods. Precision refers to 
typical fractional distance uncertainty per galaxy. Systematics level is qualitative. 
Depth indicates typical maximum distance with current samples.}
\end{table*}