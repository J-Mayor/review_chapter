\begin{table*}
\centering
\caption{Major Tully–Fisher and Related Peculiar Velocity Surveys}
\label{tab:survey_comparison}
\begin{tabular}{llcccll}
\hline
Survey & Year & Sample Size & Bands & Depth & Velocity & Reference \\
       &      &             &       & (km/s) & Measure  &           \\
\hline
SFI & 1997 & ~1300 & I-band & cz < 6000 km/s & HI linewidth (W₂₀) & \citet{giovanelli1997sfi} \\
HIPASS TF & 2006 & ~4300 HI detections & Optical + HI & cz < 12700 km/s & HI linewidth (W₅₀) & \citet{koribalski2004hipass} \\
ALFALFA & 2011 & ~31000 HI sources & Optical + HI & cz < 18000 km/s & HI linewidth (W₅₀) & \citet{haynes2011alfalfa} \\
2MTF & 2012 & ~2000 & J, H, K (2MASS) & cz < 10000 km/s & HI linewidth & \citet{hong2012tf} \\
6dFGSv & 2012 & ~8500 (FP) & K-band & cz < 16000 km/s & Velocity dispersion (FP) & \citet{springob20126dfgsv} \\
SFI++ & 2016 & ~4800 & I-band & cz < 15000 km/s & HI linewidth (W₂₀) & \citet{springob2016sfi++} \\
Cosmicflows-3 & 2016 & ~18000 distances & Multi-band compilation & cz < 30000 km/s & Multiple methods & \citet{tully2016cosmicflows3} \\
\hline
\end{tabular}
\tablecomments{Summary of major TF-based and related peculiar velocity surveys. 
Sample sizes refer to galaxies with TF or FP distance estimates. Velocity measures 
indicate the primary observable used (HI linewidth W₂₀ or W₅₀, or velocity dispersion for FP). 
Depths are approximate maximum recession velocities.}
\end{table*}